\documentclass[acmtog]{acmart}
\usepackage[english,ngerman]{babel}
\usepackage[utf8]{inputenc}

\copyrightyear{2024}
\acmYear{2024}
\citestyle{acmauthoryear}

\usepackage[figurename=Fig.]{caption}
\setcopyright{none}
\makeatletter
\renewcommand{\fnum@figure}{Abb. \thefigure}
\makeatother
\addto\captionsngerman{\renewcommand{\figurename}{Abb.}}
\settopmatter{printacmref=false} % Removes citation information below abstract
\renewcommand\footnotetextcopyrightpermission[1]{} % removes footnote with conference information in first column

\begin{document}

\title{Requirements Engineering: User Stories und Epics in Vorgehensmodellen}

\author{Jonas Pohl}
\authornote{Alle Studierenden trugen zu gleichen Teilen zu dieser Arbeit bei.}
\author{Mose Schmiedel}
\authornotemark[1]
\author{Antonia Swiridoff}
\authornotemark[1]
\affiliation{%
	\institution{Hochschule für Technik, Wirtschaft und Kultur Leipzig (HTWK Leipzig)}
	\streetaddress{Karl-Liebknecht-Str. 132}
	\city{Leipzig}
	\country{Deutschland}
	\postcode{04277}
}

\renewcommand{\shortauthors}{Swiridoff, Pohl und Schmiedel}


\begin{abstract}
\end{abstract}

\maketitle


\section{Einleitung}

\begin{itemize}
\item Motivation
\item Ziel der Arbeit
\item (Bedeutung von Requirements Engineering (RE) in der Softwareentwicklung und die wachsende Relevanz agiler Ansätze.)
\item Zielsetzung: Untersuchung der Rolle von User Stories und Epics in agilen Vorgehensmodellen und deren Einfluss auf Kommunikation und Anpassungsfähigkeit.
\item Struktur des Papers: Kurzbeschreibung der Inhalte der einzelnen Kapitel.
\end{itemize}

\section{Requirements Engineering allgemein (kurz)}
\begin{itemize}
	\item Definition und Einordnung: RE als Teilbereich des Software Engineerings.
	\item Arten von Anforderungen: Funktionale und nicht-funktionale Anforderungen, Anforderungsquellen.
	\item Ziele und Aufgaben: Klärung von Anforderungen, Sicherstellung von Qualität und Kundenzufriedenheit.
\end{itemize}

\section{Epics und User Stories}
\begin{itemize}
	\item Was sind User Stories und Epics?
	\item Unterschied zw. User Stories und klassischen Anforderungen.
	\item Rolle der Akzeptanzkriterien.
	\item Definition und Zweck (grobe Anforderungen, die später verfeinert werden).
	\item Hierarchie (Epic -> User Story -> Task).
	\item User Stories: Struktur (As a [user], I want to [goal], so that [benefit]).
	\item User Story und Epic klar voneinander trennen, was unterscheidet sie voneinander?
				(Beispiel, was später fortgesetzt wird?)
	\item Was macht gute User Story / gutes Epic aus.
\end{itemize}

\section{Autom. Verarbeitung von User Storys am Beispiel von Cucumber}

\section{Agile Vorgehensmodelle}

\begin{itemize}
	\item Besonderheiten im agilen Kontext: Herausforderungen und Anpassungen durch das Agile Manifesto und dessen Prinzipien (allgemein auf das agile Manifest nochmal eingehen).
	\item Kanban / Scrum / FDD für unterschiedliche Teamgrößen
\end{itemize}

\subsection{Kanban}
	\begin{itemize}
		\item Wie sind Epics/ User Storys in den Arbeitsprozess eingegliedert?
		\item Kommunikation im Team: User Stories und Epics als Kommunikationsmittel und zur Schaffung einer einheitlichen Sprache (Einfluss auf Kommunikation mit Kunden)
		\item Nutzen? geeignet? Vorteile? Risiken?
	\end{itemize}

\subsection{Scrum}
	\begin{itemize}
		\item Wie sind Epics/ User Storys in den Arbeitsprozess eingegliedert?
		\item Kommunikation im Team: User Stories und Epics als Kommunikationsmittel und zur Schaffung einer einheitlichen Sprache (Einfluss auf Kommunikation mit Kunden)
		\item \item Nutzen? geeignet? Vorteile? Risiken?
	\end{itemize}

\subsection{FDD}
	\begin{itemize}
		\item Wie sind Epics/ User Storys in den Arbeitsprozess eingegliedert?
		\item Kommunikation im Team: User Stories und Epics als Kommunikationsmittel und zur Schaffung einer einheitlichen Sprache (Einfluss auf Kommunikation mit Kunden)
		\item Nutzen? geeignet? Vorteile? Risiken?
	\end{itemize}



\section{Diskussion}

\begin{itemize}
	\item Vorteile? Nutzen?
	\item Risiken?
	\item Probleme bei der Anwendung von User Stories und Epics: unklare Anforderungen, Gefahr der Über- oder Unterpriorisierung.
	\item aufzeigen wie User Story und Epic helfen um agil arbeiten zu können
\end{itemize}

\section{Zusammenfassung und Ausblick}

\begin{itemize}
	\item Zusammenfassung: User Stories und Epics als Mittel zur Förderung von effektiver Kommunikation und Anpassungsfähigkeit.
	\item Ausblick:
	\begin{itemize}
		\item Potenziale durch KI-gestützte Tools zur automatischen Erstellung oder Analyse von User Stories.
	\end{itemize}
\end{itemize}


\bibliographystyle{ACM-Reference-Format}
\bibliography{01-Requirements-Engineering}

\appendix

\section{Anhang 1}

\subsection{Übungsaufgaben}

\subsection{Part Two}

Etiam commodo feugiat nisl pulvinar pellentesque. Etiam auctor sodales
ligula, non varius nibh pulvinar semper. Suspendisse nec lectus non
ipsum convallis congue hendrerit vitae sapien. Donec at laoreet
eros. Vivamus non purus placerat, scelerisque diam eu, cursus
ante. Etiam aliquam tortor auctor efficitur mattis.

\section{Anhang 2}

Nam id fermentum dui. Suspendisse sagittis tortor a nulla mollis, in
pulvinar ex pretium. Sed interdum orci quis metus euismod, et sagittis
enim maximus. Vestibulum gravida massa ut felis suscipit
congue. Quisque mattis elit a risus ultrices commodo venenatis eget
dui. Etiam sagittis eleifend elementum.

Nam interdum magna at lectus dignissim, ac dignissim lorem
rhoncus. Maecenas eu arcu ac neque placerat aliquam. Nunc pulvinar
massa et mattis lacinia.

\section{Anhang 3}

Damit es im Literaturverzeichnis angezeigt wird:\\
\cite{erbert08}, \cite{sommerville16}, \cite{balzert09}, \cite{wiegers13},
\cite{pohl15}, \cite{martin13}

\end{document}
\endinput
