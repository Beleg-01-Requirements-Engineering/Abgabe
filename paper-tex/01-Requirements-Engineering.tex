%% The first command in your LaTeX source must be the \documentclass command.
\documentclass[acmtog]{acmart}
\usepackage[english,ngerman]{babel}
\usepackage[utf8]{inputenc}

%% \BibTeX command to typeset BibTeX logo in the docs
\AtBeginDocument{%
  \providecommand\BibTeX{{%
    \normalfont B\kern-0.5em{\scshape i\kern-0.25em b}\kern-0.8em\TeX}}}

\copyrightyear{2024}
\acmYear{2024}
\citestyle{acmauthoryear}

\usepackage[figurename=Fig.]{caption}
\setcopyright{none}
\makeatletter
\renewcommand{\fnum@figure}{Abb. \thefigure}
\makeatother
\addto\captionsngerman{\renewcommand{\figurename}{Abb.}}
\settopmatter{printacmref=false} % Removes citation information below abstract
\renewcommand\footnotetextcopyrightpermission[1]{} % removes footnote with conference information in first column

%%
%% end of the preamble, start of the body of the document source.
\begin{document}

%%
%% The "title" command has an optional parameter,
%% allowing the author to define a "short title" to be used in page headers.
\title{Requirements Engineering: User Stories und Epics in Vorgehensmodellen}

%%
%% The "author" command and its associated commands are used to define
%% the authors and their affiliations.
%% Of note is the shared affiliation of the first two authors, and the
%% "authornote" and "authornotemark" commands
%% used to denote shared contribution to the research.
\author{Antonia Swiridoff}
\authornote{Alle Studierenden trugen zu gleichen Teilen zu dieser Arbeit bei.}
\author{Jonas Pohl}
\authornotemark[1]
\author{Mose Schmiedel}
\authornotemark[1]
\affiliation{%
	\institution{Hochschule für Technik, Wirtschaft und Kultur Leipzig (HTWK Leipzig)}
	\streetaddress{Karl-Liebknecht-Str. 132}
	\city{Leipzig}
	%\state{Ohio}
	\country{Deutschland}
	\postcode{04277}
}
%%
%% By default, the full list of authors will be used in the page
%% headers. Often, this list is too long, and will overlap
%% other information printed in the page headers. This
%command allows
%% the author to define a more concise list
%% of authors' names for this purpose.
\renewcommand{\shortauthors}{Swiridoff, Pohl und Schmiedel}

%%
%% The abstract is a short summary of the work to be presented in the
%% article.
\begin{abstract}
\end{abstract}

\maketitle

%% The next two lines define the bibliography style to be used, and
%% the bibliography file.
\bibliographystyle{ACM-Reference-Format}
\bibliography{01-Requirements-Engineering}

%%
%% If your work has an appendix, this is the place to put it.
\appendix

\section{Anhang 1}

\subsection{Übungsaufgaben}

\subsection{Part Two}

Etiam commodo feugiat nisl pulvinar pellentesque. Etiam auctor sodales
ligula, non varius nibh pulvinar semper. Suspendisse nec lectus non
ipsum convallis congue hendrerit vitae sapien. Donec at laoreet
eros. Vivamus non purus placerat, scelerisque diam eu, cursus
ante. Etiam aliquam tortor auctor efficitur mattis.

\section{Anhang 2}

Nam id fermentum dui. Suspendisse sagittis tortor a nulla mollis, in
pulvinar ex pretium. Sed interdum orci quis metus euismod, et sagittis
enim maximus. Vestibulum gravida massa ut felis suscipit
congue. Quisque mattis elit a risus ultrices commodo venenatis eget
dui. Etiam sagittis eleifend elementum.

Nam interdum magna at lectus dignissim, ac dignissim lorem
rhoncus. Maecenas eu arcu ac neque placerat aliquam. Nunc pulvinar
massa et mattis lacinia.

\section{Anhang 3}

Damit es im Literaturverzeichnis angezeigt wird:\\
\cite{erbert08}, \cite{sommerville16}, \cite{balzert09}, \cite{wiegers13},
\cite{pohl15}, \cite{martin13}

\end{document}
\endinput
%%
%% End of file `sample-acmtog.tex'.
