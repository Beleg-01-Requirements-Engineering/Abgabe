\documentclass[acmtog]{acmart}
\usepackage[ngerman, english]{babel}
\usepackage[utf8]{inputenc}
\usepackage[T1]{fontenc}
\usepackage{lmodern}

\copyrightyear{2024}
\acmYear{2024}
\citestyle{acmauthoryear}

\usepackage[figurename=Fig.]{caption}
\setcopyright{none}
\makeatletter
\renewcommand{\fnum@figure}{Abb. \thefigure}
\makeatother
\addto\captionsngerman{\renewcommand{\figurename}{Abb.}}
\settopmatter{printacmref=false} % Removes citation information below abstract
\renewcommand\footnotetextcopyrightpermission[1]{} % removes footnote with conference information in first column

\begin{document}

\title{Requirements Engineering: User Stories und Epics in Vorgehensmodellen}

\author{Jonas Pohl}
\authornote{Alle Studierenden trugen zu gleichen Teilen zu dieser Arbeit bei.}
\author{Mose Schmiedel}
\authornotemark[1]
\author{Antonia Swiridoff}
\authornotemark[1]
\affiliation{%
	\institution{Hochschule für Technik, Wirtschaft und Kultur Leipzig (HTWK Leipzig)}
	\streetaddress{Karl-Liebknecht-Str. 132}
	\city{Leipzig}
	\country{Deutschland}
	\postcode{04277}
}

\renewcommand{\shortauthors}{Swiridoff, Pohl und Schmiedel}


\begin{abstract}
\end{abstract}

\maketitle


\section{Einleitung}

\begin{itemize}
	\item Motivation
	\item Ziel der Arbeit
	\item (Bedeutung von Requirements Engineering (RE) in der Softwareentwicklung und die wachsende Relevanz agiler Ansätze.)
	\item Zielsetzung: Untersuchung der Rolle von User Stories und Epics in agilen Vorgehensmodellen und deren Einfluss auf Kommunikation und Anpassungsfähigkeit.
	\item Struktur des Papers: Kurzbeschreibung der Inhalte der einzelnen Kapitel.
\end{itemize}

\section{Requirements Engineering allgemein (kurz)}
\begin{itemize}
	\item Definition und Einordnung: RE als Teilbereich des Software Engineerings.
	\item Arten von Anforderungen: Funktionale und nicht-funktionale Anforderungen, Anforderungsquellen.
	\item Ziele und Aufgaben: Klärung von Anforderungen, Sicherstellung von Qualität und Kundenzufriedenheit.
\end{itemize}

\section{Epics und User Stories}
Wie dem \emph{Agile Manifesto} \cite{beedle_principles_2001} festgehalten, muss eine agile Softwareentwicklung
auf sich verändernde Anforderungen reagieren können.
Dies führt dazu, dass viele herkömmliche Methoden zur Anforderungsanalyse für diese Art von Softwareentwicklung
nicht mehr geeignet sind.

Um mit diesen Veränderungen im Requirements Engineering umzugehen, sind neue Werkzeuge und Konzepte
für diese Phase der Softwareentwicklung entstanden, welche die Prinzipien der agilen Softwareentwicklung
berücksichtigen und unterstützen.

Das Konzept von \emph{Epics und User Stories} gehört zu diesen neuen Methoden und soll das Entwicklerteam
während bei der Anforderungsanalyse unterstützen. Im folgenden Abschnitt werden die zugrundeliegenden Definitionen
dieses Konzeptes vorgestellt, anhand eines Beispiels verdeutlicht und die Vor- und Nachteile beleuchtet.

\subsection{Definition ``Epic''}
TODO
\begin{itemize}
	\item Was sind User Stories und Epics?
	\item Unterschied zw. User Stories und klassischen Anforderungen.
	\item Rolle der Akzeptanzkriterien.
	\item Definition und Zweck (grobe Anforderungen, die später verfeinert werden).
	\item Hierarchie (Epic -> User Story -> Task).
	\item User Stories: Struktur (As a [user], I want to [goal], so that [benefit]).
	\item User Story und Epic klar voneinander trennen, was unterscheidet sie voneinander?
	      (Beispiel, was später fortgesetzt wird?)
	\item Was macht gute User Story / gutes Epic aus.
\end{itemize}

\subsection{Definition ``User Story''}
\cite[p. 4]{cohn_user_2004} beschreibt ``[e]ine \emph{User Story} [als] eine Funktionalität, welche wertvoll für einen Nutzer [...] eines Systems
oder Software ist''.
Diese Funktionalität wird dabei mit ein bis zwei Sätzen in folgender Struktur formuliert:

\vspace{1em}
\texttt{As a \underline{<type of user>}, I want to \underline{<goal>}\\
	\hspace*{13em} so that \underline{<achieved value>}.}
\begin{flushright}
	\cite[p.]{balzert09}
\end{flushright}


\vspace{.5em}

\verb|<Type of user>|, \verb|<goal>| und \verb|<achieved value>| stellen hierbei Platzhalter dar,
welche je nach Anforderung ausgefüllt werden müssen.
\cite{jeffries_essential_2001} unterteilt eine User Story in folgende drei Teile:
\begin{description}
	\item[Card] repräsentiert die Anforderung, strukturiert nach oben genannter Vorlage.
	\item[Conversation] findet als (verbale) Kommunikation der Anforderung zwischen Kunde zu Entwickler statt.
	\item[Confirmation] besteht aus den Akzeptanztests, welche die notwendigen Eigenschaften der Anforderung festlegen.
\end{description}

Dabei sei hervorgehoben, dass die Card nicht etwa als vollständige Spezifikation für die geforderte Funktionalität dient,
sondern lediglich als Erinnerung und Grundlage für spätere Diskussion.
Diese wird in der Conversation durchgeführt und gegebenenfalls dokumentiert.
Ziel der Conversation ist es dabei die Details der Funktionalität zu klären
und durch Akzeptanztests in der Confirmation als Spezifikation für die Implementierung
der Funktionalität festzuhalten \cite[p. 4]{cohn_user_2004}.

\subsection{Beispiel}
Um das Konzept der \emph{Epics und User Stories} besser zu verdeutlichen, wurden alle Beispiele in diesem
Artikel für ein imaginäres Kursverwaltungssystem erstellt.
Dieses System soll Mitgliedern einer Hochschule die digitalen Verwaltung von Kursen und deren zugehörigen Informationen
und Prüfungen ermöglichen.
Außerdem sollen Studierende in der Lage sein, sich zu Kursen, bzw. Prüfungen an- und abzumelden.

Für dieses System wird in diesem Abschnitt am Beispiel eines spezifischen Epics der Umgang mit Epics und User Stories erläutert.

Das Epic ist basierend auf der Vorlage aus dem vorhergehenden Abschnitt formuliert.
Dabei sind die Phrasen, welche für die Platzhalter eingesetzt wurden, unterstrichen.

Das gewählte Epic lautet wie folgt:

\vspace{1em}
\texttt{Als \underline{Professor} möchte ich \underline{meine Kurse digital }\\
	\hspace*{3em}\underline{ verwalten}, damit \underline{die Studierenden }\\
	\hspace*{4em} \underline{ unabhängig mit diesen interagieren können}.}
\vspace{1em}

Wie schon in der Definition des Epics angesprochen handelt es sich bei einem Epic um ein umfangreiches
Arbeitspaket, welches mehr eine Vision oder größeres Ziel ausdrückt, als eine einzelne Anforderung.
Im Beispiel ist dies unschwer durch den Gebrauch von groben Formulierungen zu erkennen. So wird zum
Beispiel lediglich festgelegt, dass ein \underline{\texttt{Professor}} seine \underline{\texttt{Kurse digital verwalten}}
möchte. Diese Formulierung lässt einen großen Spielraum für die spätere Implementierung und erzwingt gleichzeitig
auch die Eingrenzung des Systems durch weitere Anforderungen, damit dieses erfolgreich modelliert und implementiert werden kann.

Einige Fragen, welche das Epic noch offen lässt sind zum Beispiel:
\begin{itemize}
	\item Was bedeutet digital verwalten?
	\item Soll eine Web-, Desktop- oder Mobile-Anwendung entwickelt werden?
	\item Welche Art von Interaktion soll für die Professoren und Studierenden möglich sein?
\end{itemize}

Die Antworten auf diese Fragen stellen weitere Anforderungen an das System dar.
Diese werden wiederum als User Stories formuliert und weiterverarbeit.
Selbstverständlich können auch hier nocheinmal User Stories entstehen, welche
zu generell sind und deshalb weitere Verfeinerung benötigen.
Tatsächlich handelt es sich dann um ein weiteres Epic, welches wiederum durch den
vorher erläuterten Prozess verfeinert werden muss.
Hierbei ist es allerdings wichtig, dass User Stories nicht bis in das kleinste Detail
aufgeteilt werden dürfen \cite[p. 6]{cohn_user_2004}.
Wie in der Definition erwähnt gehört zu einer User Story neben der \emph{Card}, der Formulierung,
auch die \emph{Conversation}, in welcher die Details der User Story geklärt werden.
Die Absicht User Stories so spezifisch wie möglich zu formulieren würde hier nur zu überflüssiger
Redundanz führen, welche schlussendlich die Effizienz des Entwicklungsteam senkt.

Anhand dieser Überlegungen sind folgende zwei User Stories als Verfeinerung des oben genannten Epics verfasst:

\vspace{1em}
\texttt{Als \underline{Professor} möchte ich \underline{die Kursverwaltung per }\\
	\hspace*{2em} \underline{ Weboberfläche bedienen können}, damit\\
	\hspace*{6em} \underline{ich von unterschiedlichen Geräten }\\
	\hspace*{14em} \underline{ darauf zugreifen kann.}}
\vspace{1em}

Diese User Story geht auf die ersten beiden Fragen ein und stellt eine in diesem Punkt
eine Spezialisierung der Anforderung dar.

Hierbei sei anzumerken, dass es ausgehend von dem Epic keine eindeutig richtige oder
falsche Spezialisierung gibt.
In diesem konkreten Fall wären alle drei Möglichkeiten,
nämlich eine Web-, Desktop- oder Mobile-Anwendung zu entwickeln, korrekt gewesen.
Das Epic lässt die Interpretation von \emph{digital} offen.

Aus diesem Grund wird in vielen Fällen für die Spezialisierung des Epics weitere Kommunikation
mit dem Kunden des System von Nöten sein.
Meist kann nur dieser die korrekte Interpretation der Anforderung liefern.

Wie und ob diese Kommunikation stattfindet ist allerdings Sache des (agilen) Vorgehensmodells, welches
das Entwicklerteam anwendet.

\vspace{1em}
\texttt{Als \underline{Professor} möchte ich \underline{die Kursinformationen }\\
	\hspace*{4.5em} \underline{ verändern können}, damit \underline{sie richtig} sind.}
\vspace{1em}

In dieser zweiten User Story findet eine Spezialisierung der dritten Frage statt.
Bei dieser Frage geht es nicht darum, die Mehrdeutigkeit eines verwendeten Begriffs zu klären,
sondern konkrete Beispiele aus einem durch das Epic aufgespannten Anforderungsbereich
zu formulieren.

So könnten in Zukunft noch weitere User Stories hinzukommen, die das Epic im Bezug auf die dritte
Frage spezialisieren.
Zum Beispiel könnte eine weitere User Story die eine mögliche Interaktion der Studierenden beschreiben:

\vspace{1em}
\texttt{Als \underline{Studierender} möchte ich \underline{mich für die Teilnahme }\\
	\hspace*{2em}\underline{ an einem Kurs registrieren },\\
	\hspace*{4em}damit \underline{mir diese Teilnahme angerechnet wird}.}
\vspace{1em}

\subsection{Was sind gute User Stories?}


\section{Automatische Verarbeitung von User Stories am Beispiel von Cucumber}

\section{Agile Vorgehensmodelle}

\begin{itemize}
	\item Besonderheiten im agilen Kontext: Herausforderungen und Anpassungen durch das Agile Manifesto und dessen Prinzipien (allgemein auf das agile Manifest nochmal eingehen).
	\item Kanban / Scrum / FDD für unterschiedliche Teamgrößen
\end{itemize}

\subsection{Kanban}
\begin{itemize}
	\item Wie sind Epics/ User Storys in den Arbeitsprozess eingegliedert?
	\item Kommunikation im Team: User Stories und Epics als Kommunikationsmittel und zur Schaffung einer einheitlichen Sprache (Einfluss auf Kommunikation mit Kunden)
	\item Nutzen? geeignet? Vorteile? Risiken?
\end{itemize}

\subsection{Scrum}
\begin{itemize}
	\item Wie sind Epics/ User Storys in den Arbeitsprozess eingegliedert?
	\item Kommunikation im Team: User Stories und Epics als Kommunikationsmittel und zur Schaffung einer einheitlichen Sprache (Einfluss auf Kommunikation mit Kunden)
	\item \item Nutzen? geeignet? Vorteile? Risiken?
\end{itemize}

\subsection{FDD}
\begin{itemize}
	\item Wie sind Epics/ User Storys in den Arbeitsprozess eingegliedert?
	\item Kommunikation im Team: User Stories und Epics als Kommunikationsmittel und zur Schaffung einer einheitlichen Sprache (Einfluss auf Kommunikation mit Kunden)
	\item Nutzen? geeignet? Vorteile? Risiken?
\end{itemize}



\section{Diskussion}

\begin{itemize}
	\item Vorteile? Nutzen?
	\item Risiken?
	\item Probleme bei der Anwendung von User Stories und Epics: unklare Anforderungen, Gefahr der Über- oder Unterpriorisierung.
	\item aufzeigen wie User Story und Epic helfen um agil arbeiten zu können
\end{itemize}

\section{Zusammenfassung und Ausblick}

\begin{itemize}
	\item Zusammenfassung: User Stories und Epics als Mittel zur Förderung von effektiver Kommunikation und Anpassungsfähigkeit.
	\item Ausblick:
	      \begin{itemize}
		      \item Potenziale durch KI-gestützte Tools zur automatischen Erstellung oder Analyse von User Stories.
	      \end{itemize}
\end{itemize}


\bibliographystyle{ACM-Reference-Format}
\bibliography{01-Requirements-Engineering}

\appendix

\section{Anhang 1}

\subsection{Übungsaufgaben}
\subsubsection{Epics und User Stories}
Ein Lebensmittelverwaltungssystem soll Privatpersonen helfen ihre gelagerten Lebensmittel abzurufen,
den Lebensmittelverbrauch zu überwachen und bei der Essensplanung unterstützen.

Finden Sie eine geeignete Anforderung für dieses System, welche einem Epic entspricht und formulieren Sie drei zugehörige
User Stories mit Hilfe der Gherkin-Grammatik!


\subsection{Part Two}

Etiam commodo feugiat nisl pulvinar pellentesque. Etiam auctor sodales
ligula, non varius nibh pulvinar semper. Suspendisse nec lectus non
ipsum convallis congue hendrerit vitae sapien. Donec at laoreet
eros. Vivamus non purus placerat, scelerisque diam eu, cursus
ante. Etiam aliquam tortor auctor efficitur mattis.

\section{Anhang 2}

Nam id fermentum dui. Suspendisse sagittis tortor a nulla mollis, in
pulvinar ex pretium. Sed interdum orci quis metus euismod, et sagittis
enim maximus. Vestibulum gravida massa ut felis suscipit
congue. Quisque mattis elit a risus ultrices commodo venenatis eget
dui. Etiam sagittis eleifend elementum.

Nam interdum magna at lectus dignissim, ac dignissim lorem
rhoncus. Maecenas eu arcu ac neque placerat aliquam. Nunc pulvinar
massa et mattis lacinia.

\section{Anhang 3}

Damit es im Literaturverzeichnis angezeigt wird:\\
\cite{erbert08}, \cite{sommerville16}, \cite{balzert09}, \cite{wiegers13},
\cite{pohl15}, \cite{martin13}

\end{document}
\endinput
